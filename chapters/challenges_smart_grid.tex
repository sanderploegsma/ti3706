\section{Challenges in the smart grid}

\subsection{Security}
\todo{Wederom een schaamteloze kopie uit \cite{mcdaniel2009security}}

Although deploying the smart grid has enormous social and technical benefits, several security and privacy concerns arise. Because grid customers are connected over a vast network of computerised meters and infrastructure, they and the infrastructure itself become vulnerable to scalable network-borne attacks. These meters make excellent targets for malicious hackers, largely because vulnerabilities can easily be monetised. Compromised meters allow users to manipulate their usage data or energy costs. Customer fraud is not the only problem that arises, the costs of patching hundreds of millions of meters also plays a big role. 

Another big issue concerning the security of smart peripherals is the increased risk of large-scale attacks on the electrical grid. Meter bots, distributed denial-of-service attacks, usage loggers, smart meter root-kits, meter-based viruses, and other malware are almost certainly in these devices’ future. 

Lastly, privacy is a key concern in the smart grid. Energy use information stored at the meter and distributed thereafter acts as an information-rich side channel, exposing customer habits and behaviours. Certain activities, such as watching television, have detectable power consumption signatures. 

To deal with these issues, a broad effort is needed to investigate smart grid security and privacy. We can’t wait to determine whether current laws and technology sufficiently protects users and utilities. This effort results in laws that will help customers, utilities and vendors to assess risk, in turn dramatically increasing smart grid adoption. We must also plan with failure in mind. Complex software systems such as these are by nature going to have bugs, which means the industry must develop comprehensive recovery strategies to overcome these difficulties. 