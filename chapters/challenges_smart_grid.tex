\section{Challenges in the smart grid} \fb{Some general explanations to the introduction.} \fb{Challenges more compressed and to the point.} \fb{Perhaps challenges should be in the introduction?}
Before a reliable smart grid can be realised, a number of challenges arise that need to be conquered. Some of these challenges are quite trivial\fb{Is this a challenge? Challenges should not be trivial} and also apply to other fields in technology. Others are quite specific to energy grids and probably need a lot more work before all the flaws are ironed out.
\todo{Improve this} \fb{Give outline of the following.}
 
\subsection{Security}
Although deploying the smart grid has enormous social and technical benefits \fb{Give outline of the following. e.g. make clear introduction}, several security and privacy concerns arise \cite{McDanielMcLaughlin2009a}. All grid users are connected using a network of sensors and two-way communication, making it an attractive target for malicious hackers. If a part of the network, such as a smart meter, is compromised, users are allowed to manipulate their usage data or energy cost. 

Apart from the demand side the grid itself is also prone to large-scale attacks, such as a Distributed Denial-of-Service (DDoS) attack. Such vulnerabilities are already being exploited in web services, although the power grid of course poses a much greater risk when compromised.

Finally, the amounts of data a smart grid requires introduces the issue of customer privacy. All energy use stored at the meter may expose certain customer behaviour, since certain activities have a significant energy footprint. You could derive, for instance, when someone is home or not - information that someone might want to stay hidden.

\todo{Conclusion}

\subsection{Demand side management}

The future smart grid needs to manage the supply and demand of energy effectively. If there is a too large discrepancy between the amount of energy that is available and the amount that is requested, either a lot of superfluous energy will be wasted, or power outages could occur.
So typically one would want to make sure the supply and demand of energy match as closely as possible at all times. This is easier said than done because not all power sources in the smart grid are constant. Wind turbines for instance yield varying amounts of energy depending on the weather conditions. The situation is further complicated by the fact that user demands also vary from time to time. In the evening hours a lot of energy is being used, but during the nighttime energy consumption is at a low. The grid however needs to be prepared for the worst, and should be able to supply everyone with energy, even during peak hours.

In the following sections multiple factors and tools that can influence the supply and demand of energy are discussed. This way the peaks in demand and the dips supply can be smoothed, and thus improve the reliability of the smart grid.

\subsubsection{Adapting human behavior}

A way to diminish the peaks in user demand, is by making the user aware of the fact, that they are creating the peaks. Theoretically speaking a peak in energy demand can be reduced (or entirely avoided) if all users spread out their energy usage over the course of the day. Practically speaking however it is very hard to change human behaviour.
\fb{What about overproduction?}

In a study where users were asked if they would be willing to reduce their energy usage by showering for a maximum of five minutes during the morning most participants reacted negatively \cite{GouldenBedwellRennick-EgglestoneEtAl2014}. They felt like the morning shower was part of their morning routine and having a device installed that would turn of the (hot) water after five minutes threatened the relaxation and comfort they experienced when taking said shower. 

More subtle approaches, such as coloured light's on outlets indicating the current electricity price using colours, or sending consumers periodic messages regarding their energy usage \cite{AyresRasemanShih2012}, did receive more positive feedback consumers. Small moments of feedback are nice for consumers, because they can choose to ignore them if they want. Unfortunately this also means that less intrusive forms of feedback are often not very effective and only reduce the total energy consumption marginally.

\subsubsection{Smart tariffs}

Humans aren’t very eager to change their energy consumption patterns on their own. They need incentives in order to be enticed to change their ways. Smart pricing is a form of such an incentive. Peak rates in wholesale energy prices are already being used \cite{SamadiMohsenian-RadSchoberEtAl2012}, however they go by unnoticed. Most end user are charged with an average price, and unaware of their energy usage during peak hours, they don’t change their consumption patterns. 

Users need to be made more aware of their energy usage during peak hours, and they need to make manual changes \cite{Mohsenian-RadLeon-Garcia2010} or energy consumption scheduling should be (partially) automated  \cite{SamadiMohsenian-RadSchoberEtAl2012}.

\subsubsection{Smart meters}

Smart meters are a modern version of classic energy meters. Apart from measuring energy usage they can also communicate with other smart meters, collect information about and control home appliances \cite{DepuruWangDevabhaktuni2011a}. With these tools the smart meter can help improve the stability of the smart grid.
\fb{Explain better how the smart meter can 'control' + the role of the user in this context.}

The smart meter is capable of temporarily decreasing the load on the grid. During peak demand hours, the meter can, for instance, delay the usage of certain home appliances, until there is more power available (at a lower price). 
This 'management of demand' can be developed even further when the smart meter is able to use batteries. A household could supply energy to the grid instead of using energy from the grid, by discharging a battery (of for instance an EV) and thus creating a power surplus for a moderate amount of time \cite{MwasiluJustoKimEtAl2014}. 

Smart meters have one security related issue though. When an household is using a smart meter, all energy consumption is logged very precisely. This data can be analyzed too reconstruct the life pattern of an entire household. Information about sleeping patterns, times consumer are at home and the amount of people living in a given household can all be derived from the information a smart meter stores \cite{Molina-MarkhamShenoyFuEtAl2010}. This privacy threat needs to be accounted for. Smart meter data should be used in clusters or aggregates in order to ensure the privacy of it's users.

\subsection{Management of Distributed Energy Rescources}
That the electricity grid is changing drastically is clear by now.\fb{Bad Style} A lot of information is being distributed to optimise the demand and the supply of energy. The demand side however is not the only place where a lot of improvement is needed. More and more types of energy sources are becoming available. Of course there are the main energy sources from the main suppliers like wind energy, the coal power plant and things like nuclear energy. But there are new sources that can help with meeting the demand during peak times. Consumers who have solar panels on their roofs or hospitals which have a backup emergency generator can help with this. \cite{Kumagai2012} Another new source can be the Electrical Vehicles consumers own. When someone arrives at home the EV is plugged in and starts charging. If they are designed to deliver energy back to the grid they can also cover for the peaks in demand and charge back up later when the energy is cheaper.\fb{repetitive intro}

The main issue with these new energy sources is that there is no easy way to handle the input of energy of these small sources. There is no interface between all of these systems. In addition to this is that all of these potential sources are usually not designed to deliver energy back to the grid. For example the backup generator of a hospital is nowadays only used during emergencies or blackouts. So the current systems have to be renewed or extended to work with the smart  grid. If the energy sources are able to communicate information to a central point the control center could manage the usage of the energy sources. 

If local DER's are used individually they might account for more problems than they solve. Better is when in a local area the DER's are grouped into Micro Grids. The Micro Grid acts as a single entity in the larger Smart Grid. It supplies the aggregated energy of its local DER's to the rest of the grid. For the management of the larger grid this could solve a lot of issues because there are less components to take into account. The directness of information flow however decreases because the route the information needs to travel is longer and has to pass more control centers. 
\todo{add additional information on micro grids}

