\section{Challenges in the smart grid}

\subsection{Security}
\todo{Wederom een schaamteloze kopie uit \cite{McDanielMcLaughlin2009a}}

Although deploying the smart grid has enormous social and technical benefits, several security and privacy concerns arise. Because grid customers are connected over a vast network of computerised meters and infrastructure, they and the infrastructure itself become vulnerable to scalable network-borne attacks. These meters make excellent targets for malicious hackers, largely because vulnerabilities can easily be monetised. Compromised meters allow users to manipulate their usage data or energy costs. Customer fraud is not the only problem that arises, the costs of patching hundreds of millions of meters also plays a big role. 

Another big issue concerning the security of smart peripherals is the increased risk of large-scale attacks on the electrical grid. Meter bots, distributed denial-of-service attacks, usage loggers, smart meter root-kits, meter-based viruses, and other malware are almost certainly in these devices’ future. 

Lastly, privacy is a key concern in the smart grid. Energy use information stored at the meter and distributed thereafter acts as an information-rich side channel, exposing customer habits and behaviours. Certain activities, such as watching television, have detectable power consumption signatures. 

To deal with these issues, a broad effort is needed to investigate smart grid security and privacy. We can’t wait to determine whether current laws and technology sufficiently protects users and utilities. This effort results in laws that will help customers, utilities and vendors to assess risk, in turn dramatically increasing smart grid adoption. We must also plan with failure in mind. Complex software systems such as these are by nature going to have bugs, which means the industry must develop comprehensive recovery strategies to overcome these difficulties. 

\subsection{Demand side management}

The future smart grid needs to manage the supply and demand of energy effectively. If there is a too large discrepancy between the amount of energy that is available and the amount that is requested, either a lot of superfluous energy will be wasted, or power outages could occur.
So typically one would want to make sure the supply and demand of energy match as closely as possible at all times. This is easier said than done because not all power sources in the smart grid are constant. Wind turbines for instance yield varying amounts of energy depending on the weather conditions. The situation is further complicated by the fact that user demands also vary from time to time. In the evening hours a lot of energy is being used, but during the nighttime energy consumption is at a low. The grid however needs to be prepared for the worst, and should be able to supply everyone with energy, even during peak hours.

In the following sections multiple factors and tools that can influence the supply and demand of energy are discussed. This way the peaks in demand and the dips supply can be smoothed, and thus improve the reliability of the smart grid.

\subsubsection{Smart users, adaptation of human behavior}

A way to manage the peaks in user demands, is by making the user aware of them. Theoretically speaking a peak in energy demand can be reduced (or entirely avoided) if all users spread out their energy usage over the course of the day. Practically speaking however it is very hard to change human behaviour.

In a study where users were asked if they would be willing to reduce their energy usage by showering for a maximum of five minutes during the morning most participants reacted negatively \cite{GouldenBedwellRennick-EgglestoneEtAl2014}. They felt like the morning shower was part of their morning routine and having a device installed that would turn of the (hot) water after five minutes threatened the relaxation and comfort they experienced when taking said shower. 

More subtle approaches, such as coloured light's on outlets indicating the current electricity price using colours, or sending consumers periodic messages regarding their energy usage \cite{AyresRasemanShih2012}, did receive more positive feedback consumers. Small moments of feedback are nice for consumers, because they can choose to ignore them if they want. Unfortunately this also means that less intrusive forms of feedback are often not very effective and only reduce the total energy consumption marginally.

\subsubsection{Smart tariffs}

Humans aren’t very eager to change their energy consumption patterns on their own. They need incentive in order to be enticed to change their ways. Smart pricing is a form of such an incentive. Peak rates in wholesale energy prices are already being used \cite{SamadiMohsenian-RadSchoberEtAl2012}, however they go by unnoticed. Most end user are charged with an average price, and unaware of their energy usage during peak hours, they don’t change their consumption patterns. 

Users need to be made more aware of their energy usage during peak hours, and they need to make manual changes \cite{Mohsenian-RadLeon-Garcia2010} or energy consumption scheduling should be (partially) automated  \cite{SamadiMohsenian-RadSchoberEtAl2012}.

\subsubsection{Smart meters, insight in enery usage and supply}

Smart meters are a modern version of the classic energy meter. Apart from measuring energy usage they can also communicate with other users of the smart grid, collect information about and control home appliances \cite{DepuruWangDevabhaktuni2011a}. With these tools the smart meter can help improve the stability of the smart grid.

The smart meter is capable of temporarily decreasing the load on the grid. During peak demand hours, the meter can, for instance, delay the usage of certain home appliances, until there is more power available (at a lower price). 
This 'management of demand' can be developed even further when the smart meter is able to use batteries. A household could supply energy to the grid instead of using energy from the grid, by discharging a battery (of for instance an EV) and thus creating a power surplus for a moderate amount of time \cite{MwasiluJustoKimEtAl2014}. 

Smart meters have one security related issue though. When an household is using a smart meter, all energy consumption is logged very precisely. This data can be analyzed too reconstruct the life pattern of an entire household. Information about sleeping patterns, times consumer are at home and the amount of people living in a given household can all be derived from the information a smart meter stores \cite{Molina-MarkhamShenoyFuEtAl2010}. This privacy threat needs to be accounted for. Smart meter data should be used in clusters or aggregates in order to ensure the privacy of it's users.

\subsection{Virtual Power Plants}
Another way the manage the load of the smart grid is to manage all the energy sources the grid contains. All new kinds of energy sources are coming available. Solar energy is growing, the number of EV's is increasing but also emergency backup generators can be used in the idea of a Virtual Power Plant(VPP). 

The VPP is the control center of the aggregation of all the different sources. It keeps track of the demand and the supply at any given time. In the demand side management the people are incentivised to lower their use during peaks. In addition to this the VPP acts like a sort of safety net if the supply cannot meet the demand, or if it gets too expensive to generate enough energy. \cite{vpprealenergy}

If the demand peak is too high the different energy sources can deliver power back to the grid. An EV can discharge its battery and charge back up later when the electricity is cheaper. The backup generators of hospitals or commercial buildings can also be used to help meet the demand. The VPP makes the use of all these different sources more manageable. Also, it can optimise the use of all the available energy resources to decrease the cost of the energy.\cite{microgridsmarketenv}
\todo{moet naar Game theory -> Micro grids}

