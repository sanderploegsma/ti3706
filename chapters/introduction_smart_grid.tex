\section{Introduction to smart grids}
The need for electricity has grown significantly since the start of the 21st century, and with the introduction of many different alternative energy sources the traditional power systems will not be able to cope with the complexity much longer. To combine all these different energy sources efficiently, a smarter system that is able to monitor manage, organise and communicate the supply and demand of electricity is needed. This system is referred to as ‘smart grid’. 

The US Department of Energy (DoE) defines a smart grid as follows \cite{doe}:
\begin{displayquote}
"Smart grid" generally refers to a class of technologies that people are using to bring utility electricity delivery systems into the 21st century, using computer-based remote control and automation. These systems are made possible by two-way digital communications technologies and computer processing that has been used for decades in other industries. They are beginning to be used on electricity networks, from the power plants and wind farms all the way to the consumers of electricity in homes and businesses. They offer many benefits to utilities and consumers -- mostly seen in big improvements in energy efficiency and reliability on the electricity grid and in energy users' homes and offices.
\end{displayquote}

\subsection{Why is a smart grid needed?}
\todo{Deze lijst verder uitwerken}
\begin{itemize}
	\item Improve power reliability and quality
	\item Minimize the need for back-up power plants (during peaks, demand side management)
	\item Enhance efficiency and capacity of existing power grids
	\item Self-healing (disruptions, outage)
	\item Accommodation of renewable and/or distributed energy sources
\end{itemize}

\todo{dit komt direct uit de referentie, moet wat uitgebreider worden uitgeschreven in eigen woorden}
The smart grid is a way to reach national energy independence, control emissions, and combat global warming. The motivation of the smart grid at the local level is somewhat more prosaic: it lets home users actively manage (and presumably reduce) their energy use, thus allowing them to become better citizens and control utility costs. From an industrial perspective, the smart grid enables time-of-use pricing (a key measure for controlling usage and reducing ceiling capacity by charging higher fees during peak hours), better capacity and usage planning, and support for more malleable energy markets. The grid controls could also enhance energy transmission management and increase resilience to system failures and cyber or physical attacks \cite{McDanielMcLaughlin2009a}.


