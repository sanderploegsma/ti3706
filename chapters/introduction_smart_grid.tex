\section{Introduction to smart grids}
The need for electricity has grown significantly since the start of the 21st century, and with the introduction of many different alternative energy sources the traditional power systems will not be able to cope with the complexity much longer. To combine all these different energy sources efficiently, a smarter system that is able to monitor manage, organise and communicate the supply and demand of electricity is needed. This system is referred to as 'smart grid'. 

The US Department of Energy (DoE) defines a smart grid as follows \cite{doe}:
\begin{quote}
"Smart grid" generally refers to a class of technologies that people are using to bring utility electricity delivery systems into the 21st century, using computer-based remote control and automation. These systems are made possible by two-way digital communications technologies and computer processing that has been used for decades in other industries. They are beginning to be used on electricity networks, from the power plants and wind farms all the way to the consumers of electricity in homes and businesses. They offer many benefits to utilities and consumers -- mostly seen in big improvements in energy efficiency and reliability on the electricity grid and in energy users' homes and offices.
\end{quote}

\subsection{Why is a smart grid needed?}
More and more alternative energy sources are incorporated into today's society, including wind farms, solar cells and generators powered by ocean tides \cite{Tromly2001}. As the current energy system is mostly powered by fossil fuels that have considerable limitations, the interest in these renewable sources is high. For instance, in time, existing fuels will not be able to keep up with our demand. Another issue with these fuels is that they are not available anywhere on the planet, so different parts of the world are heavily dependent on each other. Finally, the burning of fossil fuels has a large impact on the environment, ensuring damaging  climate changes if we keep up much longer with the current systems \cite{friedman2008hot}. 

The smart grid can accommodate renewable sources in such a way that their potential fluctuation in supply - due to cloudy weather, for instance - can be compensated using traditional sources to ensure a reliable power grid \todo{citation needed}. 

Another issue with the current system is the need for back-up power plants during peak hours. By managing the demand side, energy use can be spread across a large period of time to reduce peaks, thus eliminating the need for said back-up systems. \todo{Thomas, do we still miss something? Need a reference??}

Should a power grid fail somewhere along the way, the power supply needs to be re-routed to bypass the failure. Where our current systems more or less behaves like dominos when something goes wrong, a smart grid contains multiple sensors and communication features, making it an excellent system to automate the handling of several issues. By continuously monitoring itself a grid can detect and resolve problems almost instantly, resulting in less outage and better overall reliability.

%The smart grid is a way to reach national energy independence, control emissions, and combat global warming. The motivation of the smart grid at the local level is somewhat more prosaic: it lets home users actively manage (and presumably reduce) their energy use, thus allowing them to become better citizens and control utility costs. From an industrial perspective, the smart grid enables time-of-use pricing (a key measure for controlling usage and reducing ceiling capacity by charging higher fees during peak hours), better capacity and usage planning, and support for more malleable energy markets. The grid controls could also enhance energy transmission management and increase resilience to system failures and cyber or physical attacks \cite{McDanielMcLaughlin2009a}.


