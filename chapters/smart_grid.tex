\section{Introduction to smart grids}

Energy demand has grown significantly since the beginning of the 21st century, and with the introduction of many renewable energy sources the traditional power systems will not be able to cope with this added complexity optimally. To combine all energy sources efficiently, a smarter system, or network of systems, is needed. This network should be able to monitor, manage, and organise the supply and demand of electricity, while communicating with it's components. This type of network of systems is referred to as 'smart grid'. The US Department of Energy (DoE) defines a smart grid as follows \cite{doe}: 
\begin{quote}
"Smart grid" generally refers to a class of technologies that people are using to bring utility electricity delivery systems into the 21st century, using computer-based remote control and automation.
\end{quote}

\subsection{Why is a smart grid needed?}
Renewable energy sources are becoming more and more popular. These sources include wind energy, solar energy and energy generated by ocean tides \cite{Tromly2001}. The current energy system is mostly powered by fossil fuels that have considerable drawbacks; among these are environmental, political and capacity issues \cite{friedman2008hot}. The smart grid can incorporate renewable sources in the energy network, and help reducing the demand for fossil fuels.

%\todo{I don't like this paragraph, yet. Improve later.} Another issue with the current system is the need for back-up power plants during peak hours. By managing the demand side, energy use can be spread across a large period of time to reduce peaks, thus eliminating the need for said back-up systems.

When the energy network fails somewhere, the power supply needs to be re-routed to bypass the failure. The current energy networks are prone to a 'domino' effect when a local power outage occurs. A smart grid contains multiple sensors and communication features, making it an excellent system to automate the handling of several issues. By continuously monitoring itself a grid can reduce outage and improve overall reliability.

While the smart grid can help with the efficient incorporation of renewable energy sources into the grid, and improving the reliability of the energy distribution network, there are also some difficulties that need to be taken into account. In the next section some of these challenges are discussed.

%The smart grid is a way to reach national energy independence, control emissions, and combat global warming. The motivation of the smart grid at the local level is somewhat more prosaic: it lets home users actively manage (and presumably reduce) their energy use, thus allowing them to become better citizens and control utility costs. From an industrial perspective, the smart grid enables time-of-use pricing (a key measure for controlling usage and reducing ceiling capacity by charging higher fees during peak hours), better capacity and usage planning, and support for more malleable energy markets. The grid controls could also enhance energy transmission management and increase resilience to system failures and cyber or physical attacks \cite{McDanielMcLaughlin2009a}.

\subsection{Challenges in the smart grid} 
Before a reliable smart grid can be realised, a number of issues arise that need to be conquered. Some of these issues are quite trivial, but others are more challenging need a lot more work before all the flaws are ironed out. Among these issues are security concerns, the surges in energy demand and supply and managing the many energy sources of the smart grid.
 
\subsubsection{Security}
When deploying a smart grid, several security and privacy concerns arise \cite{McDanielMcLaughlin2009a}. All grid users are connected using a network of sensors and two-way communication, making it an attractive target for malicious hackers. If a part of the network, such as a smart meter, is compromised, users are allowed to manipulate their usage data or energy cost. The grid itself is also prone to large-scale attacks, such as a DDoS attack. \fb{cite}

Finally, the amounts of data a smart grid requires introduces the issue of customer privacy. All energy use stored at the meter may expose certain customer behaviour, since certain activities have a significant energy footprint. You could derive, for instance, when someone is home or not \cite{Molina-MarkhamShenoyFuEtAl2010}.

\subsubsection{Demand side management}

The smart grid needs to manage the supply and demand of energy effectively. If there is a too large discrepancy between the amount of energy that is available and the amount that is requested, either a lot of superfluous energy will be wasted, or power outages could occur.
So typically one would want to make sure the supply and demand of energy match as closely as possible at all times. This is easier said than done because not all power sources are constant. Solar panels for instance yield varying amounts of energy depending on the time of day. The situation is further complicated by the fact that user demands also vary from time to time. In the evening hours a lot of energy is being used, but during the nighttime energy consumption is at a low.

In the following sections multiple factors and tools that can influence the supply and demand of energy are discussed. This way the peaks in demand and the dips supply can be smoothed, and thus improve the reliability of the smart grid.

\paragraph{Adapting human behavior}

A way to diminish the peaks and dips in user demand, is by making the user aware of the fact, that they are creating the them. Theoretically speaking surges in energy demand can be reduced (or entirely avoided) if all users spread out their energy usage over the course of the day. Practically speaking however it is very hard to change human behaviour.

In a study where users were asked if they would be willing to reduce their energy usage by showering for a maximum of five minutes during the morning most participants reacted negatively \cite{GouldenBedwellRennick-EgglestoneEtAl2014}. More subtle approaches, such as coloured light's on outlets indicating the current electricity price, or sending consumers periodic messages regarding their energy usage \cite{AyresRasemanShih2012}, did receive more positive feedback. Unfortunately, less intrusive forms of feedback are often not very effective and only reduce the total energy consumption marginally.

%They felt like the morning shower was part of their morning routine and having a device installed that would turn of the (hot) water after five minutes threatened the relaxation and comfort they experienced when taking said shower.%

\paragraph{Smart tariffs}

Humans aren’t very eager to change their energy consumption patterns on their own. They need incentives in order to be enticed to change their ways. Smart pricing is a form of such an incentive. Peak rates in wholesale energy prices are already being used \cite{SamadiMohsenian-RadSchoberEtAl2012}, however they go by unnoticed. Most end user are charged with an average price, and unaware of their energy usage during peak hours, they don’t change their consumption patterns. Users need to be made more aware of their energy usage during peak hours, and they need to make manual changes \cite{Mohsenian-RadLeon-Garcia2010} or energy consumption scheduling should be (partially) automated  \cite{SamadiMohsenian-RadSchoberEtAl2012}.

\paragraph{Smart meters}

Smart meters are a modern version of classic energy meters. Apart from measuring energy usage they can also communicate with other smart meters, collect information about and control home appliances. This means a consumer might activate the washing machine, but the meters decides when it is best to start the machine, looking at current and predicted energy supply and tariffs  \cite{DepuruWangDevabhaktuni2011a}. 

This way the smart meter is capable of temporarily decreasing the load on the grid. This 'management of demand' can be developed even further when the smart meter is able to use batteries. A household could supply energy to the grid instead of using energy from the grid, by discharging a battery (of for instance an EV) and thus creating a power surplus for a moderate amount of time \cite{MwasiluJustoKimEtAl2014}. 

\subsubsection{Management of Distributed Energy Resources}
In a smart grid a lot of information is being distributed to optimise the demand and the supply of energy. More and more types of energy sources are being used. There are large energy sources, such as wind energy parks, fossil fuel power plants and nuclear power plants. Added to this list are consumers who have solar panels on their roofs, hospitals which have a backup emergency generator \cite{Kumagai2012}, and Electrical Vehicles consumers own. When someone arrives at home the EV is plugged in and starts charging, but it can also return energy to the grid in cooperation with for instance a smart meter.

The main issue with these new energy sources is that there is no easy way to handle the input of energy of these small sources without them being 'smart'.The current systems have to be altered to work within the smart grid. If the energy sources are enabled to communicate information to a central point, the control center could manage the usage of the energy sources. However, if all of these Distributed Energy Resources are used individually they might account for more problems than they solve. Better is when in a local area the DER's are grouped into Micro Grids. The Micro Grid acts as a single entity in the larger Smart Grid. It supplies the aggregated energy of its local DER's to the rest of the grid. For the management of the larger grid this could solve a lot of issues because there are less components to take into account. The directness of information flow however decreases because the route the information needs to travel is longer and has to pass more control centers. 