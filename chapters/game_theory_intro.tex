\section{Introduction to Game Theory}
As seen in the previous chapters technical challenges arise with the introduction of smart grids. Game theory, in combination with Multi Agent Systems (MAS), plays a significant role in the development of smart grids. Game theory is used together with agent-based techniques to optimally allocate re-sources among involved actors, such as renewable energy sources \cite{nguyen2013game}. Game theory concepts provide us with a language to formulate, structure, analyse and understand strategic scenarios.


\subsection{Prisoner's Dilemma}
One of the most famous examples that illustrates a situation that can be modelled by game theory is the Prisoner's Dilemma. This is an example which was formalised by Poundstone \cite{poundstone}: 

\begin{quote}
Two members of a criminal gang are arrested and imprisoned. Each prisoner is in solitary confinement with no means of communicating with the other. The prosecutors lack sufficient evidence to convict the pair on the principal charge. They hope to get both sentenced to a year in prison on a lesser charge. Simultaneously, the prosecutors offer each prisoner a bargain. Each prisoner is given the opportunity either to: betray the other by testifying that the other committed the crime, or to cooperate with the other by remaining silent. The offer is:
	\begin{itemize}
		\item If A and B each betray the other, each of them serves 2 years in prison
		\item If A betrays B but B remains silent, A will be set free and B will serve 3 years in prison (and vice versa)
		\item If A and B both remain silent, both of them will only serve 1 year in prison (on the lesser charge)
	\end{itemize}	
\end{quote}

This example shows that two individuals might not cooperate, even if its the best strategy for both. The essence of the problem is that, if the prisoners would choose a strategy that is best for both, they would both remain silent. If a prisoner only would take his own interest into account, he should betray the other player. Whatever the other player would choose (betray or remain silent), the prisoner would benefit by betraying the other person.

Game theory can provide the necessary techniques to solve problems such as the Prisoner's Dilemma. A well known solution concept is the pursue to find a Nash equilibrium, this is the state in which no player can benefit by changing its strategy while the other players keeps theirs unchanged. In this example it would be the situation where both A and B betray each other.

\subsection{Cooperative vs Noncooperative game theory}
We divide game theory into two main branches: noncooperative game theory and cooperative game theory. Noncooperative game theory is used in the strategic decision-making processes of a number of independent entities that have partially or totally conflicting interest over the outcome of a decision process that is affected by their actions\cite{keypaper}. In cooperative game theory, it is possible to enforce binding agreements between entities and strategies can be shared. 

It is important to note that noncooperative doesn't always imply that the players do not cooperate, it means that cooperations that arises must be self-enforcing with no communication or coordination of strategic choices among the players.\cite{keypaper}
