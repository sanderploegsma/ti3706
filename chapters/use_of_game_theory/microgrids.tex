\subsection{Microgrids \& Game theory}
Dividing the smart-grid into subgrids is a logical thing to do, as current electricity grids, while they're just a one-way communication, are already split up. [Needs citations] See the "verdeelkastjes" which can be seen in every neighbourhood. Besides having a great structure with less risks [citation needed], dividing smart-grids into sub-grids and eventually microgrids %todo Explain the differences between sub-grids and microgrids or remove
has even more advantages. %todo  Citations needed!

%todo Tell something about the energy efficiency

Also, when using microgrids, a game theory called \emph{a dynamic game of perfect information} %todo Explain what this game method is
 can be used to create an equillibrium in microgrids. Because the number of players is limited, as well for the demand side as the supply side within a microgrid, it is less complex and timeconsuming to calculate the best options considering each players wishes. Then, a price can be determined by a Aggregator, who can decide wether to buy or sell electricity for the microgrid \cite{MicrogridModellingPetrosAristidou}


Energy is lost during the process of redistribution through the distribution lines
VPP