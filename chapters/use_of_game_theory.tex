\section{Game Theory applied to Smart Grids}
When researching Game Theoretic approaches to manage and run Smart Grids, one will find out that a distinction can be made for different kind of subproblems arising in smart-grids research.\cite{keypaper}. In this paper the focus is on microgrids and demand side management. Both very interesting subsets of smartgrids, each with it's own challenges and smart solutions. First the microgrids are discussed, which can help tackle complex challenges and solve efficiency issues. Secondly the Demand side management (DMS) is discussed. Demand side Management will help in locating and solving issues involving the end user. 


\subsection{Microgrids \& Game theory}
Dividing the smart-grid into subgrids is a logical thing to do, as current electricity grids, while they're just a one-way communication, are already split up. [Needs citations] See the "verdeelkastjes" which can be seen in every neighbourhood. Besides having a great structure with less risks [citation needed], dividing smart-grids into sub-grids and eventually microgrids \todo{ Explain the differences between sub-grids and microgrids or remove}
has even more advantages. \todo{Citations needed!}

%todo Tell something about the energy efficiency

Also, when using microgrids, a game theory called \emph{a dynamic game of perfect information} %todo Explain what this game method is
 can be used to create an equillibrium in microgrids. Because the number of players is limited, as well for the demand side as the supply side within a microgrid, it is less complex and timeconsuming to calculate the best options considering each players wishes. Then, a price can be determined by a Aggregator, who can decide wether to buy or sell electricity for the microgrid \cite{MicrogridModellingPetrosAristidou}


Energy is lost during the process of redistribution through the distribution lines
VPP
\subsection{Demand-side management \& Game Theory}

PHEV’s
Different “games” you can “play”
Bayesian games
Markov games
Congestion games -> load balancing
\subsection{What is still missing/not possible?}
