\section{Demand-side management}
Another challenge in the design of smart grids is managing the demand of the consumer. Demand-side management (DSM) focuses on two main problems: reducing the total consumption of energy and shifting demand during peak hours \cite{keypaper}. Additionally, the power grid is currently specifically designed to be able to cope with peak rather than average demand to achieve high levels of reliability. As a result, it causes under utilisation of the power system \cite{MaDengSongEtAl2014}. Utilising game theory, solutions to implementing an efficient scheme for these problems, have been proposed in \cite{SamadiMohsenian-RadSchoberEtAl2012, SamadiSchoberWong2011, MaDengSongEtAl2014, MaharjanZhuZhangEtAl2013, ChenKishoreSnyder2011, ChenLiLowEtAl2010, Mohsenian-RadWongJatskevichEtAl2010a, SalinasLiLi2013, CaronKesidis2010, DepuruWangDevabhaktuni2011a}. These solutions range from devising smart pricing schemes, shaping behaviour of consumers, scheduling schemes for energy consumption. In this section, we will discuss the proposed solutions respectively. Out of many proposed solutions, we consider these to be showing very promising results for future research and integration with the smart grid. \todo{Perhaps add own judgment of what approach we maybe would find best - do this after we have added the other/additional subsections}.

\subsection{Smart pricing}
One way to manage the demand side is by designing smart tariffs. Designing these smart tariffs can be done through pricing methods such as Real-Time-Pricing (RTP), adaptive pricing and peak load pricing \cite{SamadiMohsenian-RadSchoberEtAl2012}. A new RTP method based on the Vickrey-Clarke-Groves (VCG) mechanism is proposed in \cite{SamadiMohsenian-RadSchoberEtAl2012}. The proposed VCG mechanism aims to charge each user based on their reported energy demand. The focus in this mechanism lies in ensuring that users will report their energy demand \textit{truthfully}. The study concludes that it can reduce the total energy consumption. Another VCG-based mechanism is proposed in \cite{SamadiSchoberWong2011}, it tries to shift the energy consumption to off-peak hours. In an auction setting as described in \cite{SamadiMohsenian-RadSchoberEtAl2012} and \cite{SamadiSchoberWong2011} it will be possible for the user to report false information and the proposed VCG-based mechanism ensures that there should be no incentive to report false information. In \cite{MaDengSongEtAl2014} a different way of ensuring \textit{truthfulness} is shown. Here is suggested to penalize users, who have  reported false information by checking their actual consumption and adding a penalty fee to their bill if there's too much of a difference between the figures. To relate the history to their payment and ensure truthfulness, an Arrow-d'Aspremont-Gerard-Varet (AGV) mechanism can be used \cite{MaDengSongEtAl2014}.
\fix{Specific use of VCG mechanism is unclear, quote [29]'s solution to shift to off-peak hours needs to be added, rewrite general structure of story telling: carefully use more connectives}

\paragraph{Demand response management}
The response the end-users have when influencing the tariffs is also called demand response management \cite{MaharjanZhuZhangEtAl2013}.
In \cite{ChenKishoreSnyder2011} it is shown that Stackelberg games can be applied to modelling the interaction between an energy management controller (i.e. the unit that controls when household appliances are scheduled) and the supplier of the energy. The energy management controller takes the role of following the leader, in this case the supplier after setting the price. Peak load and saving money for the consumer can be achieved by using RTP. Another application of Stackelberg games in DSM is the modelling of interactions between the supplier and the consumers \cite{MaharjanZhuZhangEtAl2013}. It can be shown that the game will be able to converge to an equilibrium and that it is possible that the payoff for both the supplier and consumer can be maximised. 

\subsection{Energy consumption scheduling}
A lot of papers did research in energy consumption scheduling. A good example is \cite{ChenLiLowEtAl2010} where two models are given to match the supply and shape the demand of energy in order to get a market equilibrium. An algorithm is given where both companies and customers jointly run the market with an iterative bidding scheme to find equilibrium price and allocation before the actual action of demand reduction \cite{ChenLiLowEtAl2010}. This is a way to achieve a market-clearing price (i.e. a price where supply equals demand).

The previous schemes about direct load control and smart pricing are all focused on the interaction between energy suppliers and individual end-users. Other studies show \cite{Mohsenian-RadWongJatskevichEtAl2010a, SalinasLiLi2013} that it can be beneficial to model the energy usage of end-users all together in an aggregate load. The algorithm to optimise the schedule for all users however is proven to be non-polynomial. \cite{CaronKesidis2010} So to minimise the cost for all the users would take too long and could not be computed in time. Instead of this, the problem can be distributed over all users. This can be done by letting the users communicate whenever it can be beneficial to coordinate their usage. That idea is modeled as a non-cooperative energy consumption game. The game also seeks to minimise the total cost of the energy used in a smart grid. In the game all schedules are known and in turn each user chooses its schedule such that it has the lowest cost possible. It then sends this new schedule to the rest of the users, who will in turn, if possible with this new information, choose a new schedule with a lower cost than before. The mechanic of this game is based on the fact that if the total load increases the cost is higher and that the cost functions are convex. Because the users profit from scheduling their energy consumption at times with the lowest load a Nash equilibrium can be established such that no single user can increase its profit any further  \cite{Mohsenian-RadWongJatskevichEtAl2010a}.

The 'players' of this game are not the actual users of the homes but rather the smart meters they have installed. If the equilibrium is reached and the meter has an optimal schedule it can autonomously enforce this schedule. It of course takes into account what appliances can and cannot be scheduled at different times. For example a fridge needs to always be turned on and a washing machine can be scheduled to run at a different time \cite{DepuruWangDevabhaktuni2011a}. 
