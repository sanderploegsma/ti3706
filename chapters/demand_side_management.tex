\section{Demand-side management}
Another key challenge in designing the Smart Grid is managing the demand of the consumer \cite{keypaper}. This plays an important role in the ability to reduce the load on the electric grid during peak hours (e.g. a time when households have a high cumulative energy demand). Furthermore, it allows utility companies (i.e. suppliers of the energy) to give incentive for consumers to shift their demand for energy to a time slot when the demand is low.  To tackle the challenge of managing the demand of consumers, many solutions have been proposed. In this section several of these solutions, which include mechanism design and (non)cooperative games, will be discussed. 

\subsection{Mechanism design}
One of many proposed solutions to address the issue of managing the demand is using mechanism design. In particular in \cite{SamadiMohsenian-RadSchoberEtAl2012} and \cite{MaDengSongEtAl2014} the authors propose a solution based on a Vickrey-Clarke-Groves (VCG) and Arrow-d'Aspremont-Gerard-Varet (AGV) mechanism respectively.

A VCG mechanism aims to maximise the social welfare \cite{ShohamLeyton-Brown2008} and in the proposed design in \cite{SamadiMohsenian-RadSchoberEtAl2012} in context of the smart grid, the consumer will be asked to report their expected energy usage. An important aspect is whether the information the user reports, will reflect their actual usage. In case of reporting an inaccurate amount of energy usage, this will be detrimental to the supplier \cite{MaDengSongEtAl2014}. An important notion within mechanism design is the notion of \textit{incentive compatibility} [citation needed]. In this context the notion states that the consumer would have no reason to report false information about their energy usage. Influencing pricing on energy \cite{SamadiMohsenian-RadSchoberEtAl2012} or applying punishment for inaccuracy \cite{MaDengSongEtAl2014} can ensure this property.

\subsection{Energy consumption scheduling}
A lot has been written about game theory applied in energy consumption scheduling. A good example is \cite{ChenLiLowEtAl2010} where two models are given to match the supply and shape the demand of energy in order to get a market equilibrium. An algorithm is given where both companies and customers jointly run the market with an iterative bidding scheme to find equilibrium price and allocation before the actual action of load shedding \cite{ChenLiLowEtAl2010}. This is a way to achieve a market-clearing price (i.e. a price where supply equals demand).

The previous schemes about direct load control and smart pricing are all focused on the interaction between energy suppliers and individual end-users. Other studies show \cite{Mohsenian-RadWongJatskevichEtAl2010a} that it can be beneficial to model the energy usage of end-users all together in an aggregate load. This can be done by letting the users communicate whenever it can be beneficial to coordinate their usage \cite{Mohsenian-RadWongJatskevichEtAl2010a}. 

\subsection{Stackelberg games}
\fix{A part off this section needs to move to chapter 3}
In a Stackelberg game there is a player who takes on the role of the \textit{leader} and another player who is the \textit{follower}. The leader will then try to enforce its strategy onto the follower. \cite{ShohamLeyton-Brown2008}.
In \cite{ChenKishoreSnyder2011} it is shown that Stackelberg games can be applied to modelling the interaction between an energy management controller (i.e. the unit that controls when household appliances are scheduled) and the supplier of the energy. The energy management controller will take the role of following the leader, in this case the supplier after setting the price. Peak load and saving money for the consumer can be achieved by using Real-time-pricing. Another application of Stackelberg games in managing the demand side can be done by creating a Stackelberg game between the supplier and the consumer \cite{MaharjanZhuZhangEtAl2013}. It can be shown that the game will be able to converge to an equilibrium and that it is possible that the payoff for both the supplier and consumer can be maximised. 


\todo{
\begin{enumerate}
\item[a)] PHEV’s
\item[b)]Different “games” you can “play”
\item[c)]Bayesian games
\item[d)]Markov games
\item[e)] Congestion games, load balancing
\end{enumerate}
}