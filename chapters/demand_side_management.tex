\section{Demand side management}\label{dsm}
Another challenge in the design of smart grids is managing the demand of the consumer. \ac{dsm} focuses on two main problems: reducing the total consumption of energy and shifting demand during peak hours \cite{keypaper}. Additionally, the power grid is currently specifically designed to be able to cope with peak rather than average demand to achieve high levels of reliability. As a result, it causes under utilisation of the power system \cite{MaDengSongEtAl2014}. Utilising game theory, solutions to implementing an efficient scheme for these problems, have been proposed in \cite{SamadiMohsenian-RadSchoberEtAl2012, SamadiSchoberWong2011, MaDengSongEtAl2014, MaharjanZhuZhangEtAl2013, ChenKishoreSnyder2011, ChenLiLowEtAl2010, Mohsenian-RadWongJatskevichEtAl2010a, SalinasLiLi2013, CaronKesidis2010, DepuruWangDevabhaktuni2011a}. These solutions range from devising smart pricing schemes, shaping behaviour of consumers and scheduling schemes for energy consumption. In this chapter we will discuss the proposed solutions respectively. Out of many proposed solutions we consider these to be showing very promising results for future research and integration with the smart grid.

\subsection{Smart pricing}
One way to manage the demand side is by designing smart tariffs. Designing these smart tariffs can be done through pricing methods such as \ac{rtp} (i.e. a higher load correlates to a higher price and vice versa), adaptive pricing and peak load pricing \cite{SamadiMohsenian-RadSchoberEtAl2012}. A new \ac{rtp} method based on the \ac{vcg} mechanism is proposed in \cite{SamadiMohsenian-RadSchoberEtAl2012}. In the proposed method tariffs are determined based on the local information obtained from users reporting their expected demand during certain time slots. The method aims to reduce total energy consumption. This is achieved by the supplier calculating the optimal consumption level for the user based on the reported energy consumption and assigning a new tariff. Because users can anticipate the impact on the tariffs and therefore possibly lie about their true energy consumption to get a lower tariff, the VCG mechanism will help to ensure that users will not have any incentive to do so by showing that the user cannot do better than reporting their demand truthfully. As a result, the proposed system is deemed efficient and shows that it lowers total energy consumption, because the utilities of the users and cost on the supplier is respectively maximised and minimised. Another \ac{vcg}-based mechanism is proposed in \cite{SamadiSchoberWong2011}. It tries to shift the energy consumption to off-peak hours. Similar to \cite{SamadiMohsenian-RadSchoberEtAl2012}, it requires users to report their demand to the supplier. However, the supplier computes the optimal consumption schedule for the user rather than the optimal level of consumption. This schedule aims to balance the total level of consumption.

In \cite{MaDengSongEtAl2014} a different way of ensuring users reporting their demand truthfully is shown. Here is suggested to penalise users, who have reported false information by checking their actual consumption and adding a penalty fee to their bill if there is too much of a difference between the figures. To relate the history to their payment and ensure truthfulness, an \ac{agv} mechanism can be used \cite{MaDengSongEtAl2014}.

\paragraph{Demand response management}
The response the end-users have when influencing the tariffs is also called demand response management \cite{MaharjanZhuZhangEtAl2013}. In \cite{ChenKishoreSnyder2011} it is shown that Stackelberg games can be applied to modelling the interaction between an energy management controller (i.e. the unit that controls when household appliances are scheduled) and the supplier of the energy. The energy management controller takes the role of following the leader, in this case the supplier after setting the price. Peak load and saving money for the consumer can be achieved by using \ac{rtp}. Another application of Stackelberg games in \ac{dsm} is the modelling of interactions between the supplier and the consumers \cite{MaharjanZhuZhangEtAl2013}. It can be shown that the game will be able to converge to an equilibrium and that it is possible that the payoff for both the supplier and consumer can be maximised. 

We have revealed that applications of smart pricing can both reduce the total level of energy consumption and shift peak load. However, we believe smart pricing may cause large price swings, when energy is either superfluous or scarce. Consumers may find the instability of the tariffs undesirable, because the more rational choice to either consume more energy when tariffs are low or less when tariffs are high may interfere with their daily life. 

\subsection{Energy consumption scheduling}
A lot of papers have presented research in energy consumption scheduling. A good example is where two models are given to match the supply and shape the demand of energy in order to get a market equilibrium \cite{ChenLiLowEtAl2010}. An algorithm is given where both companies and customers jointly run the market with an iterative bidding scheme to find an equilibrium price and allocation before the actual action of demand reduction \cite{ChenLiLowEtAl2010}. This is a way to achieve a market-clearing price (i.e. a price where supply equals demand).

The previous schemes about direct load control and smart pricing are all focused on the interaction between energy suppliers and individual end-users. Other studies show that it can be beneficial to model the energy usage of end-users all together in an aggregate load \cite{Mohsenian-RadWongJatskevichEtAl2010a, SalinasLiLi2013, ZhuTangLambotharanEtAl2011}. The algorithm to optimise the schedule for all users however is proven to be non-polynomial \cite{CaronKesidis2010}. So to minimise the cost for all the users would take too long and could not be computed in time. Instead of this, the problem can be distributed over all users. This can be done by letting the users communicate whenever it can be beneficial to coordinate their usage. That idea is modelled as a non-cooperative congestion game. The goal of the game is of course to minimise the total cost of the energy used in a smart grid. In the game all schedules are known and in turn each user chooses its schedule such that it has the lowest cost possible. It then sends this new schedule to the rest of the users, who will in turn, if possible with this new information, choose a new schedule with a lower cost than before. The mechanic of this game is based on the fact that if the total load increases the cost is higher and that the cost functions are convex. Because the users profit from scheduling their energy consumption at times with the lowest load a Nash equilibrium can be established such that no single user can increase its profit any further \cite{Mohsenian-RadWongJatskevichEtAl2010a, ZhuTangLambotharanEtAl2011, IbarsNavarroGiupponi2010}.

The \emph{players} of this game are not the actual users of the homes but rather the smart meters they have installed. If the equilibrium is reached and the meter has an optimal schedule it can autonomously enforce this schedule. It of course takes into account what appliances can and cannot be scheduled at different times. For example a fridge needs to always be turned on and a washing machine can be scheduled to run at a different time \cite{DepuruWangDevabhaktuni2011a}. 

\paragraph{Energy Storage}
In addition to the original energy consumption scheduling game an improved game can be applied which includes the use of energy storage devices like \acp{ev} or even dedicated batteries. This new game based on the congestion game basically works the same. So the users take turns to improve its schedule to minimise the cost of the energy used. The resources a user has however are different. A user with an \ac{ev} can discharge the battery and lower its demand on the power grid. The \ac{ev} can be recharged again later. This way the demand of the user is shifted to off-peak hours \cite{NguyenSongHan2012}. The batteries can also be used to deliver energy back to the grid to help meet demand. Users can profit from this by recharging the battery when energy is cheaper. When many users sell power to the grid the energy prices will vary every day and schedules can only be computed the day before \cite{VytelingumVoiceRamchurnEtAl2010}. In \cite{VytelingumVoiceRamchurnEtAl2010} a self learning mechanism is proposed that predicts market prices based on market trends. The user then adapts its schedule based on these predictions and updates it when actual prices are known.
