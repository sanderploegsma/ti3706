\section{Demand-side management}
Another key challenge in designing the Smart Grid is managing the demand of the consumer \cite{keypaper}. To achieve high levels of reliability, the power grid is specifically designed to be able to cope with peak rather than average demand. As a result, it causes under utilisation of the power system \cite{MaDengSongEtAl2014}. Demand side management (DSM) plays an important role in the ability to reduce the load on the electric grid during peak hours (e.g. a time when households have a high cumulative energy demand). Furthermore, it allows utility companies (i.e. suppliers of the energy) to give incentive for consumers to shift their demand for energy to a time slot when the demand is low. To tackle the challenge of managing the demand of consumers, many solutions have been proposed. In this section several solutions that use game theoretic approaches will be discussed. 

\subsection{Smart pricing}
One way to manage the demand side is by designing smart tariffs. Designing these smart tariffs can be done through pricing methods such as Real-Time-Pricing (RTP), adaptive pricing and peak load pricing \cite{SamadiMohsenian-RadSchoberEtAl2012}. A new RTP method based on the Vickrey-Clarke-Groves (VCG) mechanism is proposed in \cite{SamadiMohsenian-RadSchoberEtAl2012}. The proposed VCG mechanism aims to charge each user based on their reported energy demand. The focus in this mechanism lies in ensuring that users will report their energy demand \textit{truthfully}. The study concludes that it can reduce the total energy consumption. Another VCG-based mechanism is proposed in \cite{SamadiSchoberWong2011}, however tries to shift the energy consumption to off-peak hours. In an auction setting as described in \cite{SamadiMohsenian-RadSchoberEtAl2012} and \cite{SamadiSchoberWong2011} it will be possible for the user to report false information and the proposed VCG-based mechanism ensures that there should be no incentive to report false information. In \cite{MaDengSongEtAl2014} a different way of ensuring \textit{truthfulness} is shown. It is suggested to penalize users, who have previously reported false information by checking the consumption history to their payment, to pay extra money. To relate the history to their payment and ensure truthfulness, an Arrow-d'Aspremont-Gerard-Varet (AGV) mechanism can be used \cite{MaDengSongEtAl2014}.

\paragraph{Demand response management}
The response the end-users have when influencing the tariffs can also be called demand response management \cite{MaharjanZhuZhangEtAl2013}.
In \cite{ChenKishoreSnyder2011} it is shown that Stackelberg games can be applied to modelling the interaction between an energy management controller (i.e. the unit that controls when household appliances are scheduled) and the supplier of the energy. The energy management controller will take the role of following the leader, in this case the supplier after setting the price. Peak load and saving money for the consumer can be achieved by using RTP. Another application of Stackelberg games in DSM is modelling the interaction between the supplier and the consumer \cite{MaharjanZhuZhangEtAl2013}. It can be shown that the game will be able to converge to an equilibrium and that it is possible that the pay-off for both the supplier and consumer can be maximised. 

\subsection{Energy consumption scheduling}
A lot has been written about game theory applied in energy consumption scheduling. A good example is \cite{ChenLiLowEtAl2010} where two models are given to match the supply and shape the demand of energy in order to get a market equilibrium. An algorithm is given where both companies and customers jointly run the market with an iterative bidding scheme to find equilibrium price and allocation before the actual action of load shedding \cite{ChenLiLowEtAl2010}. This is a way to achieve a market-clearing price (i.e. a price where supply equals demand).

The previous schemes about direct load control and smart pricing are all focused on the interaction between energy suppliers and individual end-users. Other studies show \cite{Mohsenian-RadWongJatskevichEtAl2010a} that it can be beneficial to model the energy usage of end-users all together in an aggregate load. This can be done by letting the users communicate whenever it can be beneficial to coordinate their usage. This idea is modeled as a noncooperative energy consumption game. It seeks to minimise the total cost of the energy used in a smart grid. In turn each user chooses its schedule such that it has the lowest cost possible. It then sends this schedule to the rest of the users which if possible with this new information also choose a new schedule with lower cost than before. Based on the fact that if the total load increases the cost is higher and that the cost functions are convex. Because the users profit from scheduling their energy consumption at times with the lowest load a nash equilibrium can be established such that no user can increase its profit any further. \cite{Mohsenian-RadLeon-Garcia2010}

The 'players' of this game are not the actual users of the homes but rather the smart meters they have installed. If the equilibrium is reached and the meter has an optimal schedule it can autonomously enforce this schedule. It of course takes into account what 
