\section{Conclusion}

In this paper we're trying to answer the question: "What are the current challenges in the design of smart grids when utilising game theoretic approaches?" In order to give a well formed answer to this question we've looked at some problems that might occur the when using smart grids, and how game theoretic methods can help solving them. When we look at challenges using smart grids several key issues pop-up. We'll address these issues now. 

\paragraph{Managing supply and Demand}
The most important task of the smart grid is evidently supplying al of it's users with sufficient energy. This is not straightforward because energy demand cannot always be met by energy production. Additionally the amount of energy that can be produced by renewable energy sources varies from time to time due to weather circumstances for instances, and the amount of energy that is requested by users fluctuates during the day. \\
There are several game theoretic approaches that can be used to resolve this problem. There are several games that target user demands and aim to spread energy demand equally over the day. \\
\todo{smart pricing}\\
One way to manage user 
\todo{energy consumption scheduling}\\
\todo{EV's}
\\
\todo{prediction - estimating power}

\paragraph{Complexity} 
\improve{Maybe something about complexity? $->$ heuristics, benaderingsalgoritmes etc? NP-Hard, EXP or P?}\\
\todo{DERs - microgrids}\\
\todo{VVPs}\\

\paragraph{Privacy \& Security}
Another key challenge is keeping the smart grid secure, it's has to be resistant to hackers and DDos attacks for instance. The privacy of prosumers needs to be ensured as well, as smart meters process a lot of personal data. These requirements can't easily be covered using game theory. But good security and privacy management are key requirements when designing and deploying a smart grid. 

\paragraph{Current status \& near future}

\improve{Maybe state something about current solutions, tests etc. For example the paper \cite{Kumagai2012}, which states a danish island bevoming totally smart gridded}

\begin{itemize}
	\item Scalability (is the game applicable on a large scale or only in a small local market?)
	\item Interoperability (how does the game do in multiple contexts, f.e. in context of DER?)
	\item Realism (is the representation of the game realistic in a real world application?)
\end{itemize}



%We have seen that most solutions solve the supply vs demand peaks in the smart grid using game theory all comes down to one thing: Communication. Communication between all players, being a citizen using electricity to power their households\todo{cites}, companies using electricity to power their businesses or even changing their core business to selling electricity \cite{Binczewski2002,Kumagai2012}, but also citizens generating power by having sun-powered roofs, citizens playing on the energy market using their EV's to buy, store and sell power to make profit and even buying storage capacity to trade in energy \todo{search the correct citation}.

We've also seen that the current technology is just by far not explored enough, while we see a lot of solutions, we also see a lot of challenges, thus possibilities, still open for research. \todo{Summarise the possibilities}

But, even if the research \fix{find the right word for "technologie/onderzoek/status"} isn't good enough to create these smart grids, we can already improve the current electricity grid, making it smarter even when using one-way communications. \cite{AgarwalCui2012}

\paragraph{Subquestions}
\begin{itemize}
	\item What are current challenges when designing smart grids?
	\item What challenges are solved by Game Theoretic approaches?
	\item What Game Theoretic approaches currently exist?
	\begin{itemize}
		\item Microgrids
		\item VPP’s
		\item Demand side management
	\end{itemize}
	\item What challenges do these approaches solve?
	\item What challenges are still open to solve?
\end{itemize}

\todo{Round up - tying loose ends}