\section{Conclusion}\label{conclusion}
\acresetall
In this paper we addressed the question: ``What are the current challenges in the design of smart grids when utilising game theoretic approaches?'' In order to give a well formed answer to this question we have looked at some problems that might occur when using smart grids, and how game theoretic methods can help solve them. When we looked at challenges using smart grids several key issues popped up. 

The most important task of the smart grid is evidently supplying all of its users with sufficient energy. This is not trivial because energy demand cannot always be met by energy production. Additionally the amount of energy that can be produced by renewable energy sources varies from time to time due to weather circumstances for instance, and the amount of energy that is requested by users fluctuates during the day. There are several games that target user demands and aim to spread energy demand equally over the day. Energy demand can be managed with adjustments to the energy tariff. A \ac{rtp} method based on the \ac{vcg} mechanism can ensure that users oblige to predict their own energy consumption and penalising them for (grave) inaccuracy in their prediction. This way users are encouraged to report their estimated energy consumption truthfully and accurately. 

Another \ac{rtp} method based on a Stackelberg game can be modelled. The load on the energy grid can be spread out with changes to the energy tariff. When all prosumers install smart meters in their homes, the meter will negotiate when energy will be used, and how much. This way the program of for instance a washing machine can be scheduled to run at a beneficial time. When \acp{ev} are incorporated into a home, the smart meter, and by extent the smart grid, can decide to use energy from the battery of an \ac{ev} in order to save expenses when tariffs are high, or even to sell at a good price to other users that need energy. The \ac{ev} can then later be charged again when the tariffs are lower. 
We think especially dynamic energy pricing and energy storage are important factors that contribute to a steady power supply in the smart grid.

The smart grid has a lot of \acp{der}, consumers and prosumers connected to it. This can be a challenge when trying to solve computational challenges, such as predicting energy consumption patterns or trying to reach an equilibrium with \ac{rtp}. When there are too many users the problem will be too complex, and cannot be solved quickly or accurately enough. 
Microgrids are a potential solution to the complexity of the computations. A number of \acp{der} and users are grouped into a smaller microgrid, and this way the macro smart grid can be divided into any number of smaller microgrids. The grids can then optimise their own energy distribution locally, with less participants. Because of the smaller number of players in a game theoretic scenario, the model can be more precise. Because of this a game with perfect information can be played.
A microgrid can operate within the macro grid as a single actor. This way it can supply or demand energy to or from the rest of the smart grid if needed.

Another simplification mechanic is the \ac{vpp}. \acp{vpp} are designed too group \acp{der} specifically and form a single (virtual) power plant. A \ac{vpp} can keep track of all its energy supply and the demand, and determine a single tariff this way. An additional benefit of grouping a lot of \acp{der} is improved reliability. Decreased energy generation by a single renewable energy source (such as wind turbines) has less impact if multiple sources are grouped.
We think that especially microgrids are an important mechanic that make a smart grid viable. Because of the \emph{divide and conquer} approach to the problem, microgrids allow difficult computational problems to be simplified, and therefore increase the maximum scale of smart grids.

There is a lot of research related to smart grids that has not been finalised yet. While there are a lot of potential solutions being proposed, they are still very much conceptual, and only small trials have successfully deployed smart grids. There are a few challenges, for instance computational complexity and appropriate infrastructure, which still remain open for research. It is still too soon to deploy full-fledged smart grids in large urbanised areas.

But even though the research and technology is, at this point, not at a far enough state to create large scale smart grids, there are other solutions that can already be implemented. The current electricity grid can already be improved, making it smarter even when using one-way communications. Additionally individual households can start using their energy more efficiently using smart meters. 

We think that \acp{ev} are potentially a good power source. There is a lot of potential in using \acp{ev} as temporary power sources when there is need for additional energy. At this point the amount of \acp{ev} is still fairly limited. But in the future, with the increasing importance of renewable energy sources, the amount of \acp{ev} will increase, and with that also their potential uses as energy batteries. 