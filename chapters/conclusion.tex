\section{Conclusion}

....

We have seen that most solutions solve the supply vs demand peaks in the smart grid using game theory all comes down to one thing: Communication. Communication between all players, being a citizen using electricity to power their households\todo{cites}, companies using electricity to power their businesses or even changing their core business to selling electricity \cite{Binczewski2002}, but also citizens generating power by having sun-powered roofs, citizens playing on the energy market using their EV's to buy, store and sell power to make profit and even buying storage capacity to trade in energy \todo{search the correct citation}.

We've also seen that the current technology is just by far not explored enough, while we see a lot of solutions, we also see a lot of challenges, thus possibilities, still open for research. \todo{Summarise the possibilities}

But, even if the research \fix{find the right word for "technologie/onderzoek/status"} isn't good enough to create these smart grids, we can already improve the current electricity grid, making it smarter even when using one-way communications. \cite{AgarwalCui2012}

...

\improve{Maybe something about complexity? -> heuristics, benaderingsalgoritmes etc? NP-Hard, EXP or P?}